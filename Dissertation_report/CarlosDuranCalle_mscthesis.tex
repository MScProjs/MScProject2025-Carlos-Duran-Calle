% The document class supplies options to control rendering of some standard
% features in the result.  The goal is for uniform style, so some attention 
% to detail is *vital* with all fields.  Each field (i.e., text inside the
% curly braces below, so the MEng text inside {MEng} for instance) should 
% take into account the following:
%
% - author name       should be formatted as "FirstName LastName"
%   (not "Initial LastName" for example),
% - supervisor name   should be formatted as "Title FirstName LastName"
%   (where Title is "Dr." or "Prof." for example),
% - degree programme  should be "BSc", "MEng", "MSci", "MSc" or "PhD",
% - dissertation title should be correctly capitalised (plus you can have
%   an optional sub-title if appropriate, or leave this field blank),
% - dissertation type should be formatted as one of the following:
%   * for the MEng degree programme either "enterprise" or "research" to
%     reflect the stream,
%   * for the MSc  degree programme "$X/Y/Z$" for a project deemed to be
%     X%, Y% and Z% of type I, II and III.
% - year              should be formatted as a 4-digit year of submission
%   (so 2014 rather than the academic year, say 2013/14 say).

\documentclass[ % the name of the author
                    author={Carlos Duran Calle},
                % the name of the supervisor
                supervisor={Dr. Felipe Campelo},
                % the degree programme
                    degree={MSc},
                % the dissertation    title (which cannot be blank)
                     title={Comparative Machine Learning Analysis for Student Dropout Prediction in a Virtual Learning Environment},
                % the dissertation subtitle (which can    be blank)
                  subtitle={Incorporating Student Engagement and Socio-Economic Features},
                % the dissertation     type
                      type={},
                % the year of submission
                      year={2025}]{dissertation}

\begin{document}

% =============================================================================

% This section simply introduces the structural guidelines.  It can clearly
% be deleted (or commented out) if you use the file as a template for your
% own dissertation: everything following it is in the correct order to use 
% as is.

%\section*{Prelude}
%\thispagestyle{empty}






% =============================================================================

% This macro creates the standard UoB title page by using information drawn
% from the document class (meaning it is vital you select the correct degree 
% title and so on).


\maketitle

% After the title page (which is a special case in that it is not numbered)
% comes the front matter or preliminaries; this macro signals the start of
% such content, meaning the pages are numbered with Roman numerals.

\frontmatter

% This macro creates the standard UoB declaration; on the printed hard-copy,
% this must be physically signed by the author in the space indicated.

\makedecl

% LaTeX automatically generates a table of contents, plus associated lists 
% of figures, tables and algorithms.  The former is a compulsory part of the
% dissertation, but if you do not require the latter they can be suppressed
% by simply commenting out the associated macro.

\tableofcontents
\listoffigures
\listoftables
\listofalgorithms
`%\lstlistoflistings

% The following sections are part of the front matter, but are not generated
% automatically by LaTeX; the use of \chapter* means they are not numbered.

% -----------------------------------------------------------------------------

\chapter*{Abstract}

{\bf A compulsory section, of at most $1$ page} 
\vspace{1cm} 

\noindent
This section should summarise the project context, aims and objectives,
and main contributions (e.g., deliverables) and achievements.  The goal is to ensure that the 
reader is clear about what the topic is, what you have done within this 
topic, {\em and}\/ {\bf what your view of the outcome is.}

Essentially 
this section is a (very) short version of what is typically covered in more depth in the first 
chapter.  If appropriate, you should include here  
a clear statement of your research hypothesis.  This will obviously differ significantly
for each project, but an example might be as follows:

\begin{quote}
My research hypothesis is that a suitable genetic algorithm will yield
more accurate results (when applied to the standard ACME data set) than 
the algorithm proposed by Jones and Smith, while also executing in less
time.
\end{quote}

\noindent
The latter aspects should (ideally) be presented as a concise, factual 
list of the main points of achievement.  Again the points will differ for each project, but 
an might be as follows:

\begin{quote}
\noindent
\begin{itemize}
\item I spent $120$ hours collecting material on and learning about the 
      Java garbage-collection sub-system. 
\item I wrote a total of $5000$ lines of {\em Python} source code, and associated orchestration scripts. 
\item I designed a new algorithm for computing the non-linear mapping 
      from A-space to B-space using a genetic algorithm.
\item I implemented a version of the algorithm proposed by Jones and 
      Smith (2010), corrected a mistake in it, and 
      compared the results with several alternatives.
\end{itemize}
\end{quote}

% -----------------------------------------------------------------------------

%\chapter*{Summary of Changes}

%{\bf A conditional section, of at most $1$ page} 
%\vspace{1cm} 
%
%If (and only if) the dissertation represents a resubmission (e.g., as the result of
%a resit), then this section is compulsory: the content should summarise all
%non-trivial changes made to the initial submission.  Otherwise you can
%omit it, since {\bf a summary of this type is only needed for resubmissions}.
%
%When included, the section will ideally be used to highlight additional
%work completed, and address criticism raised in any associated feedback.
%Clearly it is difficult to give generic advice about how to do so, but
%an example might be as follows:
%
%\begin{quote}
%\noindent
%\begin{itemize}
%\item Feedback from the initial submission criticised the design and 
%      implementation of my genetic algorithm, stating ``there seems 
%      to have been no attention to computational complexity during the
%      design, and obvious methods of optimisation are missing within
%      the resulting implementation''.  Chapter 3 now includes a
%      comprehensive analysis of the algorithm, in terms of both time
%      and space.  While I have not altered the algorithm itself, I
%      have included a cache mechanism (also detailed in Chapter 3)
%      that provides a significant improvement in average run-time.
%\item I added a feature in my implementation to allow automatic rather
%      than manual selection of various parameters; the experimental
%      results in Chapter 4 have been updated to reflect this.
%\item Questions after the presentation highlighted a range of related
%      work that I had not considered: I have make a number of updates 
%      to Chapter 2, resolving this issue.
%\end{itemize}
%\end{quote}

% -----------------------------------------------------------------------------

\chapter*{Supporting Technologies}

{\bf A compulsory section, of at most $1$ page}
\vspace{1cm} 

\noindent
This section should present a detailed summary, in bullet point form, 
of any third-party resources (e.g., hardware and software components) 
used during the project.  Use of such resources is always perfectly 
acceptable: the goal of this section is simply to be clear about how
and where they are used, so that a clear assessment of your work can
result.  The content can focus on the project topic itself (rather,
for example, than including ``I used \mbox{\LaTeX} to prepare my 
dissertation''); an example is as follows:

\begin{quote}
\noindent
\begin{itemize}

\item I used the {\em Pandas} and {\em Seaborn} public-domian Python Libraries. 

\item I used a parts of the OpenCV computer vision library to capture 
      images from a camera, and for various standard operations (e.g., 
      threshold, edge detection).

\item I used Amazon Web Services for remote storage and processing of data. Specifically, I used:
 \begin{itemize}
 \item Simple Storage Service (S3) for data storage
 \item Elastic Compute Cloud (EC2) for provision of virtual machines
 \item Elastic Beanstalk for scaling and load management
 \item Sagemaker for all the machine learning components of my project. 
 \end{itemize}
 
\item I used \LaTeX\ to format my thesis, via the desktop service {\em TeXstudio}. 
\end{itemize}
\end{quote}

% -----------------------------------------------------------------------------

\chapter*{Notation and Acronyms}

%{\bf An optional section, of roughly $1$ or $2$ pages}
\vspace{1cm} 

%\noindent
%Any well written document will introduce notation and acronyms before
%their use, {\em even if} they are standard in some way: this ensures 
%any reader can understand the resulting self-contained content.  
%
%Said introduction can exist within the dissertation itself, wherever 
%that is appropriate.  For an acronym, this is typically achieved at 
%the first point of use via ``Advanced Encryption Standard (AES)'' or 
%similar, noting the capitalisation of relevant letters.  However, it 
%can be useful to include an additional, dedicated list at the start 
%of the dissertation; the advantage of doing so is that you cannot 
%mistakenly use an acronym before defining it.  A limited example is 
%as follows:

\begin{quote}
\noindent
\begin{tabular}{lcl}
SDP                 &:     & Student Dropout Predictor	\\
MOOCs               &:     & Massive Open Online Courses	\\
VLEs                &:     & Virtual Learning Environments	\\
OULAD               &:     & Open University Learning Analytics Dataset	\\
ML                  &:     & Machine Learning	\\
RF                  &:     & Random Forest	\\
LG                  &:     & Logistic Regression	\\
LightGBM            &:     & Light Gradient-Boosting Machine	\\
NNs                 &:     & Neural Networks	\\
SE                  &:     & Student Engagement	\\
KNN                 &:     & K-Nearest Neighbors	\\
KNN                 &:     & K-Nearest Neighbors	\\
KNN                 &:     & K-Nearest Neighbors	\\
KNN                 &:     & K-Nearest Neighbors	\\
KNN                 &:     & K-Nearest Neighbors	\\
KNN                 &:     & K-Nearest Neighbors	\\
%                    &\vdots&                                                                      \\
%${\mathcal H}( x )$ &:     & the Hamming weight of $x$                                            \\
%${\mathbb  F}_q$    &:     & a finite field with $q$ elements                                     \\
$x_i$               &:     & the $i$-th bit of some binary sequence $x$, st. $x_i \in \{ 0, 1 \}$ \\
\end{tabular}
\end{quote}

% -----------------------------------------------------------------------------

\chapter*{Acknowledgements}

{\bf An optional section, of at most $1$ page}
\vspace{1cm} 

\noindent
It is common practice (although totally optional) to acknowledge any
third-party advice, contribution or influence you have found useful
during your work.  Examples include support from friends or family, 
the input of your Supervisor and/or Advisor, external organisations 
or persons who  have supplied resources of some kind (e.g., funding, 
advice or time), and so on.

\vspace{1cm}
Dave Cliff writes here to say huge thanks to his colleague Dr Dan Page for sharing this \LaTeX\ thesis template, which was originally written by Dan, for Computer Science dissertations. Dave edited Dan's original to better suit the needs of the Data Science MSc: please don't hassle Dan about any of this, but do feel free to contact Dave if you have any questions or comments on it.  

% =============================================================================

% After the front matter comes a number of chapters; under each chapter,
% sections, subsections and even subsubsections are permissible.  The
% pages in this part are numbered with Arabic numerals.  Note that:
%
% - A reference point can be marked using \label{XXX}, and then later
%   referred to via \ref{XXX}; for example Chapter\ref{chap:context}.
% - The chapters are presented here in one file; this can become hard
%   to manage.  An alternative is to save the content in seprate files
%   the use \input{XXX} to import it, which acts like the #include
%   directive in C.

\mainmatter

% -----------------------------------------------------------------------------

\chapter{Introduction}
\label{chap:introduction}

%{\bf A compulsory chapter, roughly 10\% of the total page-count}
\vspace{1cm} 

% putting a \noindent before the first para in each chapter looks nicer.
\noindent
%This chapter should describe the project context, and motivate each of
%the proposed aims and objectives.  Ideally, it is written at a fairly 
%high-level, and easily understood by a reader who is technically 
%competent but not an expert in the topic itself.
%
%In short, the goal is to answer three questions for the reader.  First, 
%what is the project topic, or problem being investigated?  Second, why 
%is the topic important, or rather why should the reader care about it?  
%For example, why there is a need for this project, who will benefit from the 
%project and in what way (e.g., clients/end-users who needed some analysis
%done, or other data scientists who might need the tools you have developed), what 
%work does the project build on and why is the selected approach either
%important and/or interesting (e.g., fills a gap in literature, applies
%results from another field to a new problem).  Finally, what are the 
%central challenges involved and why are they significant? 
The rise of online learning environments, such as Massive Open Online Courses (MOOCs) and Virtual Learning Environments (VLEs), has transformed the educational landscape, providing unprecedented access to education for diverse populations \cite{maiz_olazabalaga_research_2016}. However, despite these advancements, a significant challenge persists: a high dropout rate among students. Research indicates that approximately $78\%$ of students enrolled in online courses do not complete their studies, primarily due to a lack of face-to-face interaction \cite{simpson_can_we_do_better}. Another study states that the main reason for withdrawal from VLEs is the lack of student engagement, followed by the inability to locate materials or activities for assessments \cite{kuzilek_ou_2015}. Therefore, student engagement is a key area of research, as low engagement negatively impacts final grades, knowledge retention, and dropout rates \cite{staikopoulos_SE_2015}. In VLEs, the more students engage in meaningful activities, the higher the probability that they will enjoy the course, perform well, and complete it \cite{jung_learning_2018}.

This project aims to address the problem by applying machine learning (ML) models, comparing them, and selecting the best one as part of the Student Dropout Predictor (SDP). For this purpose, data from the Open University Learning Analytics Dataset (OULAD) is used \cite{kuzilek_OULAD_2017}. The Open University, the largest in the UK, offers over a thousand online courses and provides full online degree programs \cite{hlosta_modellingVLE_2018}.

This alarming statistic underscores the urgent need to develop effective strategies for early prediction and mitigation of student dropout, enabling timely support that can help learners stay engaged and successfully complete their courses.

\section{ML Models for Predictions}
ML, a part of artificial intelligence, allows computer programs to automatically find complex patterns in features taken from existing data. This helps in making smart decisions about new data \cite{holland_ML_1992}. ML algorithms are trained on sample data and later evaluated with unseen data \cite{kotsiantis_2004}. This type of ML, where the algorithm learns a mapping function from input variables (features) to an output variable (label) using labelled training data, is known as Supervised Learning \cite{murphy_ml_2012}. ML algorithms can provide instructors with real-time insights about students, enabling early interventions during the course \cite{kai_2017}. ML is widely applied to develop predictive models from student data, handling both numerical and categorical variables to effectively model student behaviours and outcomes \cite{baker_educational_2014}. 

This project conducts a comparative evaluation of six ML models: Random Forest (RF), Logistic Regression (LR), K-Nearest Neighbors (KNN), LightGBM, Support Vector Machine (SVM), and Neural Networks (NNs). Each model is assessed on its ability to classify student outcomes into the discrete labels ‘Withdrawn’ (0), ‘Fail’ (1), or ‘Pass’ (2), with particular emphasis on accurately identifying the ‘Withdrawn’ class. For example, RF is known for its robustness and ability to handle large datasets with high dimensionality \cite{breiman_rf_2001}, while LightGBM is optimized for speed and efficiency, making it suitable for large-scale applications \cite{friedman_gbm_2001}. LR offers interpretable results \cite{harrell_LR_2015}, KNN captures local data patterns, SVM performs well in high-dimensional spaces \cite{cortes_svm_1995}, and NNs can model complex relationships \cite{lecun_nn_gradient-applied_1998}. Supervised learning algorithms can be divided into classifiers and regressors, where the previously ML models mentioned, when used as classifiers, serve to predict discrete class labels \cite{geron_oreilly_2019}.

The primary evaluation metric will be recall for the 'Withdrawn' class (0), as this reflects the model’s ability to correctly identify students who have dropped out. Maximizing recall for this minority class is critical to ensure timely support and intervention. Given the imbalanced nature of the classification problem, focusing on dropout recall helps prevent overlooking students who need immediate assistance \cite{sokolova_classification_tasks_2009}. The comparative analysis aims to identify which model achieves the highest recall for the 'Withdrawn' class, thereby determining the most effective approach for dropout detection.

\section{Research Objectives}
Using data from the Open University Learning Analytics Dataset (OULAD), the study will evaluate how well each model detects withdrawal cases, with a focus on maximizing recall for this class. Additionally, the project incorporates a 'Student Engagement' variable \cite{hussain_student_engagement_prediction_2018} and socio-economic factors such as the deprivation index to enrich the analysis of factors influencing student withdrawal.

Student engagement will be measured as a simple yes/no (binary) variable, following the method described by Mushtaq et al. \cite{hussain_student_engagement_prediction_2018}. This variable will be calculated using information such as scores on course assignments, whether the student passed or failed the course, and how active the student was in the virtual learning environment (VLE).

%\paragraph{}
\noindent
The specific objectives of this project are:
\begin{itemize}
	\item To incorporate the student engagement feature as a meaningful predictor for the machine learning models.
	\item To identify which socio-demographic variables are meaningfully linked to student dropout.
	\item To select a predictive model capable of identifying students likely to withdraw at early stages of their course.
\end{itemize}

Earlier studies that used OULAD data, such as those by Tomasevic et al. \cite{tomasevic_comparison_supervised_data_2020} and Hussain et al. \cite{hussain_student_engagement_prediction_2018}, have looked at student engagement and predicted student dropout. However, these studies did not fully examine which demographic variables showed stronger connections with dropout. The main reason for this project is to provide useful insights into the importance of demographic variables by improving the student dropout model with the addition of student engagement data. Additionally, this project will include the timing of assessments to help make predictions at an early stage of the course. Adding this analysis aims to offer another way to improve models that predict student dropout. For this project, the prediction results will be grouped into three classes: 'Withdrawal' (0), 'Fail' (1), or 'Pass' (2).

\section{Significance and Contributions}
The importance of this project extends beyond academic curiosity; it has practical implications for educators, administrators, and policymakers. By accurately predicting student dropout, institutions can implement timely interventions, tailor support services, and improve course designs, ultimately leading to higher retention rates and better educational outcomes \cite{kahu_student_engagement_2013}. Additionally, the insights gained from this research will benefit data scientists and educational technologists by providing a framework for developing predictive tools that can be applied across various educational contexts.

This project builds on existing literature that has explored student engagement and dropout prediction using machine learning techniques \cite{tomasevic_comparison_supervised_data_2020, hussain_student_engagement_prediction_2018}. However, it seeks to fill a critical gap by integrating demographic variables and engagement metrics into a comprehensive predictive model. The selected approach is significant as it not only addresses the immediate problem of dropout prediction but also contributes to the broader discourse on enhancing student engagement in online learning environments.

\section{Challenges}
Central challenges in this endeavour include ensuring the accuracy and reliability of the predictive models, addressing potential biases in the data, and effectively measuring student engagement at various stages of the learning process. These challenges are significant as they directly impact the validity of the predictions and the effectiveness of subsequent interventions. By tackling these issues, this project aims to provide valuable insights that can inform future research and practice in the field of online education.

\paragraph{}
%\noindent
In summary, this project focuses on performing and comparing six machine learning models to identify the best-performing model for detecting students who have withdrawn from online courses. By emphasizing recall for the 'Withdrawn' class and incorporating engagement and socio-economic factors, the study aims to provide actionable insights that support timely interventions and improve student retention in virtual learning environments.
 
%The chapter should conclude with a concise bullet point list that 
%summarises the aims, objectives, {\bf and achievements}\/ of your work. 

% -----------------------------------------------------------------------------

\chapter{Technical Background}
\label{chap:background}

{\bf A compulsory chapter, roughly 20\% of the total page-count}
\vspace{1cm} 

\noindent
This chapter is intended to describe the technical basis on which execution
of the project depends.  The goal is to provide a detailed explanation of
the specific problem at hand, and existing work that is relevant such as an
existing algorithm that you use, alternative solutions proposed, supporting
technologies, and relevant literature. The literature you include should cover 
appropriate peer-reviewed academic publications and books, and maybe also 
news and current-affairs articles published in reputable sources such as 
{\em The Economist} magazine or {\em The Financial Times} newspaper. 

Your thesis should include complete set of references/bibliography: everything that you cite should be in there, in full. This means including the publisher's name for anything that is a book; the editors and title of books that a paper appears in as one item in a larger collection (e.g. proceedings volumes and/or thematic edited collections) and the relevant page-numbers; and so on. Put most simply, we require and expect your references to look like the references in a published peer-reviewed academic paper, so you can work out whether you still have work to do by comparing your references to the reference-list on the publications that you're working from. 

Note there is a subtle difference from
this and a full-blown literature {\em survey}.  The latter might try
to capture and organise (e.g., categorise somehow) {\em all}\/ related work,
potentially offering meta-analysis, whereas here the goal is simply to
ensure that your thesis is self-contained.  Put another way, after reading 
this chapter an intelligent non-expert reader with no prior knowledge of your project should have obtained enough background to 
understand what {\em you}\/ have done (by reading subsequent sections), and then 
accurately assess your work.  You might view an additional goal as giving 
the reader confidence that you are able to absorb, understand and clearly 
communicate highly technical material.

Just as there is no single ideal structure for an MSc thesis, there is no one correct name for this chapter. You could just call it {\em Background}\/ if you wish, or {\em Technical Background}. Or if you feel you have a lot to say you could split this chapter into two: you might have your first three chapters being:
\begin{enumerate}
\item Introduction
\item Context
\item Related Work
\end{enumerate}
\noindent
But if you prefer to use a more concise structure, you could instead use:
\begin{enumerate}
\item Introduction: Contextual Background
\item Technical Background
\end{enumerate}
\noindent
The choice is yours. 


The key thing is that by the end of these first two or three chapters you have told the reader everything they need to know so that they can understand the rest of your thesis. This is particularly important because at least one of the people who actually examines your thesis will be a UoB academic, one of our lecturers or professors, who you can assume knows almost nothing about the specifics of what work you have done, and who is given a copy of your thesis to mark. So what you write has to adequately explain things to that examiner. You can safely assume that whoever examines your thesis is numerate, intelligent, understands programming and data analytics etc, but you cannot safely assume that their specific individual expertise is a perfect match to the topic of your thesis (it's in that sense that the examiner is a {\em non-expert}). So, the person you write for, your {\em audience}, should be that undefined group of academics who might possibly examine your thesis: if you write for that audience and do a good job, your thesis should be understandable by a wide range of people, including potential employers and colleagues in the world of work.

% -----------------------------------------------------------------------------

\chapter{Execution}
\label{chap:execution}

{\bf A topic-specific chapter, roughly 30\% of the total page-count}
\vspace{1cm} 

\noindent
This chapter is intended to describe what you did: the goal is to explain
the main activity or activities, of any type, which constituted your work 
during the project.  The content is highly topic-specific. For some 
projects it will make sense to split the content into two main sections, or maybe even into two separate chapters: one 
will discuss the design of something, including any rationale or decisions made, 
and the other will discuss how this design was realised via some form of 
implementation.  You could instead give this chapter the title ``Design and Implementation"; or you might split this content into two chapters, one titled ``Design" and the other ``Implementation" .

Note that it is common to include evidence of ``best practice'' project 
management (e.g., use of version control, choice of programming language 
and so on).  Rather than simply a rote list, make sure any such content 
is useful and/or informative in some way: for example, if there was a 
decision to be made then explain the trade-offs and implications 
involved.

\section{Example Section}

This is an example section; 
the following content is auto-generated dummy text.
\lipsum

\subsection{Example Sub-section}

\begin{figure}[t]
\centering
foo
\caption{This is an example figure.}
\label{fig}
\end{figure}

\begin{table}[t]
\centering
\begin{tabular}{|cc|c|}
\hline
foo      & bar      & baz      \\
\hline
$0     $ & $0     $ & $0     $ \\
$1     $ & $1     $ & $1     $ \\
$\vdots$ & $\vdots$ & $\vdots$ \\
$9     $ & $9     $ & $9     $ \\
\hline
\end{tabular}
\caption{This is an example table.}
\label{tab}
\end{table}

\begin{algorithm}[t]
\For{$i=0$ {\bf upto} $n$}{
  $t_i \leftarrow 0$\;
}
\caption{This is an example algorithm.}
\label{alg}
\end{algorithm}

\begin{lstlisting}[float={t},caption={This is an example listing.},label={lst},language=C]
for( i = 0; i < n; i++ ) {
  t[ i ] = 0;
}
\end{lstlisting}

This is an example sub-section;
the following content is auto-generated dummy text.
Notice the examples in Figure~\ref{fig}, Table~\ref{tab}, Algorithm~\ref{alg}
and Listing~\ref{lst}.
\lipsum

\subsubsection{Example Sub-sub-section}

This is an example sub-sub-section;
the following content is auto-generated dummy text.
\lipsum

\paragraph{Example paragraph.}

This is an example paragraph; note the trailing full-stop in the title,
which is intended to ensure it does not run into the text.

% -----------------------------------------------------------------------------

\chapter{Critical Evaluation}
\label{chap:evaluation}

{\bf A topic-specific chapter, roughly 30\% of the total page-count} 
\vspace{1cm} 

\noindent
This chapter is intended to evaluate what you did.  The content is highly 
topic-specific, but for many projects will have flavours of the following:

\begin{enumerate}
\item functional  testing, including analysis and explanation of failure 
      cases,
\item behavioural testing, often including analysis of any results that 
      draw some form of conclusion wrt. the aims and objectives,
      and
\item evaluation of options and decisions within the project, and/or a
      comparison with alternatives.
\end{enumerate}

\noindent
This chapter often acts to differentiate project quality: even if the work
completed is of a high technical quality, critical yet objective evaluation 
and comparison of the outcomes is crucial.  In essence, the reader wants to
learn something, so the worst examples amount to simple statements of fact 
(e.g., ``graph X shows the result is Y''); the best examples are analytical 
and exploratory (e.g., ``graph X shows the result is Y, which means Z; this 
contradicts [1], which may be because I use a different assumption'').  As 
such, both positive {\em and}\/ negative outcomes are valid {\em if} presented 
in a suitable manner.

% -----------------------------------------------------------------------------

\chapter{Conclusion}
\label{chap:conclusion}

{\bf A compulsory chapter,  roughly 10\% of the total page-count}
\vspace{1cm} 

\noindent
The concluding chapter(s) of a dissertation are often underutilized because they're 
too often left too close to the deadline: it is important to allocate enough time and 
attention to closing off the story, the narrative, of your thesis.

Again, there is no single correct way of closing a thesis. 

One good way of doing this is to have a single chapter consisting of three parts:

\begin{enumerate}
\item (Re)summarise the main contributions and achievements, in essence
      summing up the content.
\item Clearly state the current project status (e.g., ``X is working, Y 
      is not'') and evaluate what has been achieved with respect to the 
      initial aims and objectives (e.g., ``I completed aim X outlined 
      previously, the evidence for this is within Chapter Y'').  There 
      is no problem including aims which were not completed, but it is 
      important to evaluate and/or justify why this is the case.
\item Outline any open problems or future plans.  Rather than treat this
      only as an exercise in what you {\em could} have done given more 
      time, try to focus on any unexplored options or interesting outcomes
      (e.g., ``my experiment for X gave counter-intuitive results, this 
      could be because Y and would form an interesting area for further 
      study'' or ``users found feature Z of my software difficult to use,
      which is obvious in hindsight but not during at design stage; to 
      resolve this, I could clearly apply the technique of Bloggs {\em et al.}.
\end{enumerate}

Alternatively, you might want to divide this content into two chapters: a penultimate chapter with a title such as ``Further Work" and then a final chapter ``Conclusions". Again, there is no hard and fast rule, we trust you to make the right decision. 

And this, the final paragraph of this thesis template, is just a bunch of citations, added to show how to generate a BibTeX bibliography. Sources that have been randomly chosen to be cited here include:
%\cite{miller_etal_2018_clojure,webber_marwan_2015,touretzky_2013_lisp,eckmann_etal_1987,marwan_2011,vach_2015,shiller_2017,vytelingum_2006,tesfatsion_2002,rust_etal_1992}.




% =============================================================================

% Finally, after the main matter, the back matter is specified.  This is
% typically populated with just the bibliography.  LaTeX deals with these
% in one of two ways, namely
%
% - inline, which roughly means the author specifies entries using the 
%   \bibitem macro and typesets them manually, or
% - using BiBTeX, which means entries are contained in a separate file
%   (which is essentially a database) then imported; this is the 
%   approach used below, with the databased being dissertation.bib.
%
% Either way, the each entry has a key (or identifier) which can be used
% in the main matter to cite it, e.g., \cite{X}, \cite[Chapter 2}{Y}.

\backmatter

%\bibliographystyle{unsrt}
\bibliography{sample_bibtex.bib}

% -----------------------------------------------------------------------------

% The dissertation concludes with a set of (optional) appendices; these are 
% the same as chapters in a sense, but once signalled as being appendices via
% the associated macro, LaTeX manages them appropriately.

\appendix

\chapter{An Example Appendix}
\label{appx:example}

Content which is not central to, but may enhance the dissertation can be 
included in one or more appendices; examples include, but are not limited
to

\begin{itemize}
\item lengthy mathematical proofs, numerical or graphical results which 
      are summarised in the main body,
\item sample or example calculations, 
      and
\item results of user studies or questionnaires.
\end{itemize}

\noindent
Note that in line with most research conferences, the examiners are not
obliged to read such appendices.

% =============================================================================

\end{document}
